\documentclass[11pt, a4paper]{article}

\usepackage[ngerman]{babel}
\usepackage{graphicx} 
\usepackage[utf8]{inputenc}
\usepackage{fancyhdr}
\usepackage{changepage}
\usepackage[onehalfspacing]{setspace}
\usepackage{ragged2e}
\usepackage{ amssymb, amsmath, amsthm, dsfont }
\usepackage[width = 18cm, top = 2.5cm, bottom = 3cm]{geometry}
\usepackage{extarrows}
\usepackage{listings,color}
\usepackage[usenames,dvipsnames,svgnames,table]{xcolor}
\usepackage[obeyspaces]{url}
\usepackage{hyperref}
% --------- Variabel, auf jedem Blatt ändern!
\newcommand{\blattnummer}{10}
\newcommand{\datum}{6. Juli 2017}
	% Punktezahlen & Summe
\newcommand{\p}{7}
\newcommand{\pp}{7}
\newcommand{\ppp}{6}
\newcommand{\pppp}{}
\newcommand{\sump}{20}
% --------- Macros

\newcommand{\myTitleString} {}
\newcommand{\myAuthorString} {}
\newcommand{\mySubTitleString} {}
\newcommand{\myDateString} {}

\newcommand{\myTitle}[1] {\renewcommand {\myTitleString}{#1}}
\newcommand{\mySubTitle}[1] {\renewcommand {\mySubTitleString}{#1}}
\newcommand{\myAuthor}[1] {\renewcommand{\myAuthorString}		{#1}}
\newcommand{\myDate}[1] {\renewcommand{\myDateString}{#1}}

\makeatletter
\newcommand*{\centerfloat}{%
  \parindent \z@
  \leftskip \z@ \@plus 1fil \@minus \textwidth
  \rightskip\leftskip
  \parfillskip \z@skip}
\makeatother

\newcommand{\makeMyTitle}
{
\pagestyle{fancy}
\fancyhead[L]
{
\begin{tabular}{l}
\myTitleString
\\ \mySubTitleString 
\\ \myDateString
\end{tabular}
} 			
\fancyhead[C]{}
\fancyhead[R]{\myAuthorString}
\fancyfoot[C]{\thepage}
}

\setlength{\headheight}{45pt}

\makeatletter
\renewcommand*\env@matrix[1][*\c@MaxMatrixCols c]{%
  \hskip -\arraycolsep
  \let\@ifnextchar\new@ifnextchar
  \array{#1}}
\makeatother

    % args: Aufgabennummer, erreichbare Punkte
\newcommand{\aufgabe}[2] {\section*{Aufgabe \blattnummer.#1 (Punkte:\qquad/#2)}}
\newcommand{\aufgabenteil}[1] {\textbf{(#1)}}
% ---------
\setlength{\parindent}{0pt}
\begin{document}

\myTitle{\textsc{Datenbanken und Informationssysteme}}
\mySubTitle{Übung \blattnummer}
\myDate{\datum}
\myAuthor
{
\begin{tabular}{l l}
359109, &Michelle Milde\\
356148, &Philipp Hochmann\\
356092, &Daniel Schleiz
\end{tabular}
}
\makeMyTitle

\hfill
\begin{tabular}{|c|c|c|c|}\hline
   1 & 2 & 3 & $\sum$\\\hline
  	 \qquad/\p & \qquad/\pp & \qquad/\ppp & \qquad/\sump\\\hline % abhängig vom Übungsblatt
 \end{tabular}
\hfill Korrigiert am:\underline{\hspace{3cm}}
\hfill
\vspace*{30pt}


\aufgabe{1}{\p}
\aufgabenteil{a}
\begin{adjustwidth}{20pt}{20pt}
\begin{itemize}
\item $conf(s_1)=\{ (w_3(x),r_2(x)),(w_2(y),w_3(y)),(w_3(y),r_2(y)), (w_3(z),w_2(z))\}$
\item $conf(s_2)=\{ (r_3(x),w_1(x)),(r_2(y),w_3(y)),(r_2(y),w_1(y)),(w_3(y),w_2(y)),(w_3(y),w_1(y)),(w_2(y),w_1(y)),$\\ $(r_2(z),w_3(z)),(r_2(z),w_1(z)),(r_3(z),w_2(z)),
	(r_3(z),w_1(z)),(w_3(z),w_2(z)),(w_3(z),w_1(z)),(w_2(z),w_1(z))\}$
\end{itemize}
\end{adjustwidth}
\aufgabenteil{b}
\begin{adjustwidth}{20pt}{20pt}
\begin{itemize}
\item $commit(s_1)=\{ t_2,t_3\}$. Somit besitzt der Konfliktgraph $G_1$ die Knoten $t_2$ und $t_3$. Da $(w_3(x),r_2(x))\in conf(s_1)$ und $(w_2(y),w_3(y))\in conf(s_1)$, existiert in $G_1$
	eine Kante von $t_2$ zu $t_3$ und umgekehrt. Da somit $G_1$ einen Kreis besitzt, ist $s_1$ \textit{nicht konfliktserialisierbar}.
\item $commit(s_2)=\{ t_1,t_2,t_3\}$. Somit besitzt der Konfliktgraph $G_2$ die Knoten $t_1,t_2$ und $t_3$. Da $(r_2(y),w_3(y))\in conf(s_2)$ und $(w_3(y),w_2(y))\in conf(s_2)$, existiert 
	in $G_2$ eine Kante von $t_2$ zu $t_3$ und umgekehrt. Da somit $G_2$ einen Kreis enthält, ist $s_2$ \textit{nicht konfliktserialisierbar}.
\end{itemize}
\end{adjustwidth}


\aufgabe{2}{\pp}
\aufgabenteil{a}
\begin{adjustwidth}{20pt}{20pt}

\end{adjustwidth}
\aufgabenteil{b}
\begin{adjustwidth}{20pt}{20pt}

\end{adjustwidth}
\aufgabenteil{c}
\begin{adjustwidth}{20pt}{20pt}

\end{adjustwidth}
\aufgabenteil{d}
\begin{adjustwidth}{20pt}{20pt}

\end{adjustwidth}



\aufgabe{3}{\ppp}
\aufgabenteil{a}
\begin{adjustwidth}{20pt}{20pt}

\end{adjustwidth}
\aufgabenteil{b}
\begin{adjustwidth}{20pt}{20pt}

\end{adjustwidth}
\aufgabenteil{c}
\begin{adjustwidth}{20pt}{20pt}

\end{adjustwidth}

% --------
\end{document}
