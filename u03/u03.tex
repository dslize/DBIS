\documentclass[11pt, a4paper]{article}

\usepackage{graphicx} 
\usepackage[utf8]{inputenc}
\usepackage{fancyhdr}
\usepackage{changepage}
\usepackage[onehalfspacing]{setspace}
\usepackage{ragged2e}
\usepackage{ amssymb, amsmath, amsthm, dsfont }
\usepackage[width = 18cm, top = 2.5cm, bottom = 3cm]{geometry}
\usepackage{extarrows}
% --------- Variabel, auf jedem Blatt ändern!
\newcommand{\blattnummer}{3}
\newcommand{\datum}{11. Mai 2017}
	% Punktezahlen & Summe
\newcommand{\p}{6}
\newcommand{\pp}{8}
\newcommand{\ppp}{6}
\newcommand{\pppp}{}
\newcommand{\sump}{20}
% --------- Macros

\newcommand{\myTitleString} {}
\newcommand{\myAuthorString} {}
\newcommand{\mySubTitleString} {}
\newcommand{\myDateString} {}

\newcommand{\myTitle}[1] {\renewcommand {\myTitleString}{#1}}
\newcommand{\mySubTitle}[1] {\renewcommand {\mySubTitleString}{#1}}
\newcommand{\myAuthor}[1] {\renewcommand{\myAuthorString}		{#1}}
\newcommand{\myDate}[1] {\renewcommand{\myDateString}{#1}}

\newcommand{\makeMyTitle}
{
\pagestyle{fancy}
\fancyhead[L]
{
\begin{tabular}{l}
\myTitleString
\\ \mySubTitleString 
\\ \myDateString
\end{tabular}
} 			
\fancyhead[C]{}
\fancyhead[R]{\myAuthorString}
\fancyfoot[C]{\thepage}
}

\setlength{\headheight}{45pt}

\makeatletter
\renewcommand*\env@matrix[1][*\c@MaxMatrixCols c]{%
  \hskip -\arraycolsep
  \let\@ifnextchar\new@ifnextchar
  \array{#1}}
\makeatother

    % args: Aufgabennummer, erreichbare Punkte
\newcommand{\aufgabe}[2] {\section*{Aufgabe \blattnummer.#1 (Punkte:\qquad/#2)}}
\newcommand{\aufgabenteil}[1] {\textbf{(#1)}}
% ---------
%\setlength{\parindent}{0pt}
\begin{document}

\myTitle{\textsc{Datenbanken und Informationssysteme}}
\mySubTitle{Übung \blattnummer}
\myDate{\datum}
\myAuthor
{
\begin{tabular}{l l}
359109, &Michelle Milde\\
356148, &Philipp Hochmann\\
356092, &Daniel Schleiz
\end{tabular}
}
\makeMyTitle

\hfill
\begin{tabular}{|c|c|c|c|}\hline
   \blattnummer.1 & \blattnummer.2 & \blattnummer.3 & $\sum$\\\hline
  	 \qquad/\p & \qquad/\pp & \qquad/\ppp & \qquad/\sump\\\hline % abhängig vom Übungsblatt
 \end{tabular}
\hfill Korrigiert am:\underline{\hspace{3cm}}
\hfill
\vspace*{30pt}


\aufgabe{1}{\p}
\aufgabenteil{a}
\begin{adjustwidth}{20pt}{20pt}
\begin{tabular}{|c c|}\hline
\multicolumn{2}{|c|}{$\pi_{\text{Name,Nummer}}(\text{Starter})$} \\\hline
Name & Nummer \\\hline
William Boing & 10 \\
William Boing & 30 \\
William Boing & 40 \\
James Bond & 50 \\
James Bond & 20 \\
Herman Gram & 60 \\
Herman Gram & 70 \\\hline
\end{tabular}
\end{adjustwidth}
\aufgabenteil{b}
\begin{adjustwidth}{20pt}{20pt}
\begin{tabular}{|c c c c|}\hline
\multicolumn{4}{|c|}{$\pi_{\text{Name,Nummer,Herzfrequenz,Distanz}}(\rho_{\text{Nummer$\leftarrow$Sieger,Wettkampf$\leftarrow$Name}}(\text{Wettkampf})\bowtie \text{Starter})$} \\\hline
Name & Nummer & Herzfrequenz & Distanz \\\hline
William Boing & 10 & 140 & 42 \\
William Boing & 40 & 149 & 21 \\\hline
\end{tabular}
\end{adjustwidth}
\aufgabenteil{c}
\begin{adjustwidth}{20pt}{20pt}
\begin{tabular}{|c c|}\hline
\multicolumn{2}{|c|}{$\pi_{\text{Nummer,Sieger}}(\rho_{\text{NameWettkampf$\leftarrow$Name}}(\text{Wettkampf})\bowtie_{\text{NameWettkampf=Wettkampf}}\text{Starter})$} \\\hline
Nummer & Sieger \\\hline
10 & 10 \\
60 & 70 \\
30 & 60 \\
40 & 40 \\
70 & 20 \\\hline
\end{tabular}
\end{adjustwidth}
\aufgabenteil{d}
\begin{adjustwidth}{20pt}{20pt}
\begin{tabular}{|c c|}\hline
\multicolumn{2}{|c|}{$\pi_{\text{Alter,Nummer}}(\sigma_{\text{Distanz}=42}(\rho_{\text{Wettkampf$\leftarrow$Name}}(\text{Wettkampf}))\bowtie(\text{Läufer$\bowtie$Starter}))$} \\\hline
Alter & Nummer \\\hline
55 & 10 \\
55 & 30 \\\hline
\end{tabular}
\end{adjustwidth}
\aufgabenteil{e}
\begin{adjustwidth}{20pt}{20pt}
\begin{tabular}{|c c c|}\hline
\multicolumn{3}{|c|}{$\pi_{\text{Name,Zeit,Distanz}}(\sigma_{\text{Alter}<12}(\text{Läufer})\bowtie\text{Starter}\bowtie\rho_{\text{Wettkampf$\leftarrow$Name}}(\text{Wettkampf}))$} \\\hline
Name & Zeit & Distanz \\\hline
Herman Gram & 4:29:06 & 21 \\
Herman Gram & 3:43:11 & 21 \\\hline
\end{tabular}
\end{adjustwidth}
\aufgabenteil{f}
\begin{adjustwidth}{20pt}{20pt}
\begin{tabular}{|c c|}\hline
\multicolumn{2}{|c|}{$\pi_{\text{Name,Herzfrequenz}}(\rho_{\text{Nummer$\leftarrow$Distanz,Wettkampf$\leftarrow$Name}}(\text{Wettkampf})\bowtie\text{Starter}\bowtie\text{Läufer})$} \\\hline
Name & Herzfrequenz \\\hline
\end{tabular}
\end{adjustwidth}


\aufgabe{2}{\pp}
\aufgabenteil{a}
\begin{adjustwidth}{20pt}{20pt}
$\pi_{\text{FName}}(\sigma_{\text{Erscheinungsjahr $>$ 2005 } \wedge \text{ Genre="Horror"} }(\text{Film}))$ 
\end{adjustwidth}
\aufgabenteil{b}
\begin{adjustwidth}{20pt}{20pt}
$\pi_{\text{FName, SName, Charakter}}(\text{Film} \bowtie \text{Spielt} \bowtie \text{Schauspieler})$
\end{adjustwidth}
\aufgabenteil{c}
\begin{adjustwidth}{20pt}{20pt}
$\pi_{\text{FID}}(\sigma_{\text{Genre="Komödie"} }(\text{Film}) \bowtie \text{Spielt} \bowtie \sigma_{\text{SName="Charles Chaplin"}}(\text{Schauspieler}))$
\end{adjustwidth}
\aufgabenteil{d}
\begin{adjustwidth}{20pt}{20pt}
$\pi_{\text{RName}}(\text{Regisseur} \bowtie \text{Dreht} \bowtie \text{Film} \bowtie \sigma_{\text{Charakter="Batman"}}(\text{Spielt}))$
\end{adjustwidth}
\aufgabenteil{e}
\begin{adjustwidth}{20pt}{20pt}
$\pi_{\text{FID, FName, Genre, Erscheinungsjahr}}(\text{Film} \bowtie \text{Spielt} \bowtie \sigma_{\text{SName} \neq \text{"Til Schweiger"}}(\text{Schauspieler}))$
\end{adjustwidth}
\aufgabenteil{f}
\begin{adjustwidth}{20pt}{20pt}
$\pi_{\text{SName}}(\rho_{\text{Er} \leftarrow \text{Erscheinungsjahr}, \text{ S} \leftarrow \text{SID}}(\text{Schauspieler} \bowtie \text{Spielt} \bowtie \text{Film}) \bowtie_{\text{Er}=\text{Er}^\prime - 1, \text{ S}=\text{S}^\prime} \newline
\rho_{\text{Er}^\prime \leftarrow \text{Erscheinungsjahr}, \text{ S}^\prime \leftarrow \text{SID}}(\text{Schauspieler} \bowtie \text{Spielt} \bowtie \text{Film}))$
\end{adjustwidth}


\aufgabe{3}{\ppp}
Kürze im Folgenden Sozialversicherungsnummer durch SVN ab.\\
\aufgabenteil{a}
\begin{adjustwidth}{20pt}{20pt}
$\{ [w.\text{Bezeichnung}, w.\text{Adresse}]\ |\ w\in \text{Wahllokal } \wedge \ wk \in \text{Wahlkreis } \wedge\ w.\text{Wahlkreisnummer } = wk.\text{Wahlkreisnummer } \wedge wk.\text{Bezeichnung } =\text{'Aachen II'}\}$
\end{adjustwidth}
\aufgabenteil{b}
\begin{adjustwidth}{20pt}{20pt}
$\{ [wb.\text{Name}]\ |\ wb \in \text{Wahlberechtiger } \wedge\ wb.\text{Geburtsjahr } < 1990 \wedge\ wa \in \text{waehlt } \wedge\ wa.\text{SVN } = wb.\text{SVN } \wedge\ wa.\text{Briefwahl } = \text{'ja'}\}$
\end{adjustwidth}
\aufgabenteil{c}
\begin{adjustwidth}{20pt}{20pt}
$\{ [wb.\text{SVN}]\ |\ wb \in \text{Wahlberechtiger } \wedge\ ((wa \in \text{waehlt } \wedge\ wa.\text{SVN } = wb.\text{SVN } \wedge\ wa.\text{Wahlkreisnummer } \neq 42) \vee\ (\forall wae \in \text{waehlt }(wae.\text{SVN } \neq wb.\text{SVN })))\}$
\end{adjustwidth}
\aufgabenteil{d}
\begin{adjustwidth}{20pt}{20pt}
$\{ [wb.\text{Name}]\ |\ wb \in \text{Wahlberechtigter } \wedge\ wa \in \text{waehlt } \wedge\ wa.\text{SVN } = wb.\text{SVN } \wedge\ wae \in \text{waehlt } \wedge\ wae.\text{SVN } = wb.\text{SVN } \wedge\ wa.\text{Wahllokal } = wae.\text{Wahllokal } \wedge\
    wa.\text{Wahljahr } \neq wae.\text{Wahljahr }\}$
\end{adjustwidth}
% --------
\end{document}
