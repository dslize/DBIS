\documentclass[11pt, a4paper]{article}

\usepackage[ngerman]{babel}
\usepackage{graphicx} 
\usepackage[utf8]{inputenc}
\usepackage{fancyhdr}
\usepackage{changepage}
\usepackage[onehalfspacing]{setspace}
\usepackage{ragged2e}
\usepackage{ amssymb, amsmath, amsthm, dsfont }
\usepackage[width = 18cm, top = 2.5cm, bottom = 3cm]{geometry}
\usepackage{extarrows}
\usepackage{listings,color}
\usepackage[usenames,dvipsnames,svgnames,table]{xcolor}
\usepackage[obeyspaces]{url}
\usepackage{hyperref}
% --------- Variabel, auf jedem Blatt ändern!
\newcommand{\blattnummer}{9}
\newcommand{\datum}{29. Juni 2017}
	% Punktezahlen & Summe
\newcommand{\p}{6}
\newcommand{\pp}{8}
\newcommand{\ppp}{6}
\newcommand{\pppp}{}
\newcommand{\sump}{20}
% --------- Macros

\newcommand{\myTitleString} {}
\newcommand{\myAuthorString} {}
\newcommand{\mySubTitleString} {}
\newcommand{\myDateString} {}

\newcommand{\myTitle}[1] {\renewcommand {\myTitleString}{#1}}
\newcommand{\mySubTitle}[1] {\renewcommand {\mySubTitleString}{#1}}
\newcommand{\myAuthor}[1] {\renewcommand{\myAuthorString}		{#1}}
\newcommand{\myDate}[1] {\renewcommand{\myDateString}{#1}}

\makeatletter
\newcommand*{\centerfloat}{%
  \parindent \z@
  \leftskip \z@ \@plus 1fil \@minus \textwidth
  \rightskip\leftskip
  \parfillskip \z@skip}
\makeatother

\newcommand{\makeMyTitle}
{
\pagestyle{fancy}
\fancyhead[L]
{
\begin{tabular}{l}
\myTitleString
\\ \mySubTitleString 
\\ \myDateString
\end{tabular}
} 			
\fancyhead[C]{}
\fancyhead[R]{\myAuthorString}
\fancyfoot[C]{\thepage}
}

\setlength{\headheight}{45pt}

\makeatletter
\renewcommand*\env@matrix[1][*\c@MaxMatrixCols c]{%
  \hskip -\arraycolsep
  \let\@ifnextchar\new@ifnextchar
  \array{#1}}
\makeatother

    % args: Aufgabennummer, erreichbare Punkte
\newcommand{\aufgabe}[2] {\section*{Aufgabe \blattnummer.#1 (Punkte:\qquad/#2)}}
\newcommand{\aufgabenteil}[1] {\textbf{(#1)}}
% ---------
%\setlength{\parindent}{0pt}
\begin{document}

\myTitle{\textsc{Datenbanken und Informationssysteme}}
\mySubTitle{Übung \blattnummer}
\myDate{\datum}
\myAuthor
{
\begin{tabular}{l l}
359109, &Michelle Milde\\
356148, &Philipp Hochmann\\
356092, &Daniel Schleiz
\end{tabular}
}
\makeMyTitle

\hfill
\begin{tabular}{|c|c|c|c|}\hline
   1 & 2 & 3 & $\sum$\\\hline
  	 \qquad/\p & \qquad/\pp & \qquad/\ppp & \qquad/\sump\\\hline % abhängig vom Übungsblatt
 \end{tabular}
\hfill Korrigiert am:\underline{\hspace{3cm}}
\hfill
\vspace*{30pt}


\aufgabe{1}{\p}



\aufgabe{2}{\pp}
\aufgabenteil{a}
\begin{adjustwidth}{20pt}{20pt}
\begin{tt}
PREFIX dbo:  <http://dbpedia.org/ontology/>\\
SELECT (COUNT(?match) AS ?c) WHERE \{\\
\null\qquad\ ?match dbo:champion <http://dbpedia.org/resource/Tiger\_Woods> .\\
\}
\end{tt}
\\ \ \\
Ergebnis:\\
\begin{tabular}{ |c|c| } 
 \hline
 &c\\ 
 \hline
 1&39\\
 \hline
\end{tabular}
\end{adjustwidth}

\aufgabenteil{b}
\begin{adjustwidth}{20pt}{20pt}
\begin{tt}
PREFIX dbo:  <http://dbpedia.org/ontology/>\\
SELECT ?uri ?height WHERE \{\\
\null\qquad\ ?uri dbo:birthPlace <http://dbpedia.org/resource/Cologne> .\\
\null\qquad\ ?uri dbo:height ?height .\\
\}\\
ORDER BY DESC(?height)\\
LIMIT 5
\end{tt}
\\ \ \\
Ergebnis:\\
\begin{tabular}{ |c|c|c| } 
 \hline
 & uri & height \\ 
 \hline
1 & http://dbpedia.org/resource/Yassin\_Idbihi & 2.08\\ 
2 & http://dbpedia.org/resource/Uwe\_Krupp & 1.9812\\ 
3 & http://dbpedia.org/resource/Thomas\_Kessler & 1.97\\ 
4 & http://dbpedia.org/resource/Lars\_Leese & 1.9558\\ 
5 & http://dbpedia.org/resource/Jan\_Frodeno & 1.94\\ 
 \hline
\end{tabular}
\end{adjustwidth}
\aufgabenteil{c}
\begin{adjustwidth}{20pt}{20pt}
\begin{tt}
PREFIX dbo:  <http://dbpedia.org/ontology/>\\
PREFIX dbp: <http://dbpedia.org/property/>\\
SELECT ?uri WHERE \{\\
\null\qquad\ ?uri dbo:ingredient <http://dbpedia.org/resource/Beer> .\\
\null\qquad\ ?uri dbp:name ?name .\\
\null\qquad\ FILTER CONTAINS(?name, "Beer")\\
\}
\end{tt}
\\ \ \\
Ergebnis:\\
\begin{tabular}{ |c|c|c| } 
 \hline
 & uri & name \\ 
 \hline
1 & http://dbpedia.org/resource/Beer\_bread & Beer bread\\ 
2 & http://dbpedia.org/resource/Beer\_cheese\_(spread) & Beer cheese\\ 
3 & http://dbpedia.org/resource/Beer\_soup & Beer soup\\
 \hline
\end{tabular}
\end{adjustwidth}



\aufgabe{3}{\ppp}
\aufgabenteil{a}
\begin{adjustwidth}{20pt}{20pt}

\end{adjustwidth}
\aufgabenteil{b}
\begin{adjustwidth}{20pt}{20pt}

\end{adjustwidth}

% --------
\end{document}
