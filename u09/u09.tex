\documentclass[11pt, a4paper]{article}

\usepackage[ngerman]{babel}
\usepackage{graphicx} 
\usepackage[utf8]{inputenc}
\usepackage{fancyhdr}
\usepackage{changepage}
\usepackage[onehalfspacing]{setspace}
\usepackage{ragged2e}
\usepackage{ amssymb, amsmath, amsthm, dsfont }
\usepackage[width = 18cm, top = 2.5cm, bottom = 3cm]{geometry}
\usepackage{extarrows}
\usepackage{listings,color}
\usepackage[usenames,dvipsnames,svgnames,table]{xcolor}
\usepackage[obeyspaces]{url}
\usepackage{hyperref}
% --------- Variabel, auf jedem Blatt ändern!
\newcommand{\blattnummer}{9}
\newcommand{\datum}{29. Juni 2017}
	% Punktezahlen & Summe
\newcommand{\p}{6}
\newcommand{\pp}{8}
\newcommand{\ppp}{6}
\newcommand{\pppp}{}
\newcommand{\sump}{20}
% --------- Macros

\newcommand{\myTitleString} {}
\newcommand{\myAuthorString} {}
\newcommand{\mySubTitleString} {}
\newcommand{\myDateString} {}

\newcommand{\myTitle}[1] {\renewcommand {\myTitleString}{#1}}
\newcommand{\mySubTitle}[1] {\renewcommand {\mySubTitleString}{#1}}
\newcommand{\myAuthor}[1] {\renewcommand{\myAuthorString}		{#1}}
\newcommand{\myDate}[1] {\renewcommand{\myDateString}{#1}}

\makeatletter
\newcommand*{\centerfloat}{%
  \parindent \z@
  \leftskip \z@ \@plus 1fil \@minus \textwidth
  \rightskip\leftskip
  \parfillskip \z@skip}
\makeatother

\newcommand{\makeMyTitle}
{
\pagestyle{fancy}
\fancyhead[L]
{
\begin{tabular}{l}
\myTitleString
\\ \mySubTitleString 
\\ \myDateString
\end{tabular}
} 			
\fancyhead[C]{}
\fancyhead[R]{\myAuthorString}
\fancyfoot[C]{\thepage}
}

\setlength{\headheight}{45pt}

\makeatletter
\renewcommand*\env@matrix[1][*\c@MaxMatrixCols c]{%
  \hskip -\arraycolsep
  \let\@ifnextchar\new@ifnextchar
  \array{#1}}
\makeatother

    % args: Aufgabennummer, erreichbare Punkte
\newcommand{\aufgabe}[2] {\section*{Aufgabe \blattnummer.#1 (Punkte:\qquad/#2)}}
\newcommand{\aufgabenteil}[1] {\textbf{(#1)}}
% ---------
\setlength{\parindent}{0pt}
\begin{document}

\myTitle{\textsc{Datenbanken und Informationssysteme}}
\mySubTitle{Übung \blattnummer}
\myDate{\datum}
\myAuthor
{
\begin{tabular}{l l}
359109, &Michelle Milde\\
356148, &Philipp Hochmann\\
356092, &Daniel Schleiz
\end{tabular}
}
\makeMyTitle

\hfill
\begin{tabular}{|c|c|c|c|}\hline
   1 & 2 & 3 & $\sum$\\\hline
  	 \qquad/\p & \qquad/\pp & \qquad/\ppp & \qquad/\sump\\\hline % abhängig vom Übungsblatt
 \end{tabular}
\hfill Korrigiert am:\underline{\hspace{3cm}}
\hfill
\vspace*{30pt}


\aufgabe{1}{\p}
\begin{adjustwidth}{20pt}{20pt}
\begin{tt}
$@$prefix rdf: <\nolinkurl{http://www.w3.org/1999/02/22-rdf-syntax-ns#}> .\\
$@$prefix rdfs: <\nolinkurl{http://www.w3.org/2000/01/rdf-schema#}> .\\
$@$prefix foaf: <\nolinkurl{http://xmlns.com/foaf/0.1/}> .\\
$@$prefix exp: <\nolinkurl{http://example.org/the-expanse/}> .\\
\ \\
exp:Marsianer rdf:type foaf:Person;\\
\null \qquad              foaf:label "Martian"@eng, "Pomang"@BL.\\
		   
exp:Cocktail foaf:name "Gin Tonic";\\
\null \qquad             exp:hatZutat \_:b1, \_:b2, \_:b3.\\
\_:b1 foaf:name "Gin";\\
\null\qquad     exp:Eigenschaft "gekühlt".\\
\_:b2 foaf:name "Tonic Water".\\
\_:b3 foaf:name "Zitronenschale".\\

\# Verwendung von Blank Nodes, da wir verschiedene Untereigenschaften und die dazu gehörenden Informationen
\# (hier Zutaten und deren Eigenschaften) einer Elterneigenschaft (hier 'hatZutat') zu-und unterordnen wollen
\# z.B. unterordnung der Zutat 'Gin', welche wieder eine eigene Eigenschaft ('gekühlt') besitzt.

exp:Planet1 foaf:name "Mars";\\
\null\qquad            rdf:type exp:Planet.\\
exp:Planet2 foaf:name "Erde";\\
\null\qquad            rdf:type exp:Planet.\\
exp:Person1 rdf:type foaf:Person;\\
\null\qquad            foaf:name "Bobbie Draper".\\
\ \\		
exp:Planet1 exp:imKrieg exp:Planet2.\\
\ \\
exp:Aussage rdf:type rdf:Statement;\\
\null\qquad            rdf:subject exp:Planet1;\\
\null\qquad            rdf:predicate exp:imKrieg;\\
\null\qquad            rdf:object exp:Planet2;\\
\null\qquad            exp:gesagtVon exp:Person1.
\end{tt}
\end{adjustwidth}


\aufgabe{2}{\pp}
\aufgabenteil{a}
\begin{adjustwidth}{20pt}{20pt}
\begin{tt}
PREFIX dbo:  <http://dbpedia.org/ontology/>\\
SELECT (COUNT(?match) AS ?c) WHERE \{\\
\null\qquad\ ?match dbo:champion <http://dbpedia.org/resource/Tiger\_Woods> .\\
\}
\end{tt}
\\ \ \\
Ergebnis:\\
\begin{tabular}{ |c|c| } 
 \hline
 &c\\ 
 \hline
 1&39\\
 \hline
\end{tabular} \ \\
\end{adjustwidth}

\aufgabenteil{b}
\begin{adjustwidth}{20pt}{20pt}
\begin{tt}
PREFIX dbo:  <http://dbpedia.org/ontology/>\\
SELECT ?uri ?height WHERE \{\\
\null\qquad\ ?uri dbo:birthPlace <http://dbpedia.org/resource/Cologne> .\\
\null\qquad\ ?uri dbo:height ?height .\\
\}\\
ORDER BY DESC(?height)\\
LIMIT 5
\end{tt}
\\ \ \\
Ergebnis:\\
\begin{tabular}{ |c|c|c| } 
 \hline
 & uri & height \\ 
 \hline
1 & http://dbpedia.org/resource/Yassin\_Idbihi & 2.08\\ 
2 & http://dbpedia.org/resource/Uwe\_Krupp & 1.9812\\ 
3 & http://dbpedia.org/resource/Thomas\_Kessler & 1.97\\ 
4 & http://dbpedia.org/resource/Lars\_Leese & 1.9558\\ 
5 & http://dbpedia.org/resource/Jan\_Frodeno & 1.94\\ 
 \hline
\end{tabular} \ \\
\end{adjustwidth}
\aufgabenteil{c}
\begin{adjustwidth}{20pt}{20pt}
\begin{tt}
PREFIX dbo:  <http://dbpedia.org/ontology/>\\
PREFIX dbp: <http://dbpedia.org/property/>\\
SELECT ?uri WHERE \{\\
\null\qquad\ ?uri dbo:ingredient <http://dbpedia.org/resource/Beer> .\\
\null\qquad\ ?uri dbp:name ?name .\\
\null\qquad\ FILTER CONTAINS(?name, "Beer")\\
\}
\end{tt}
\\ \ \\
Ergebnis:\\
\begin{tabular}{ |c|c|c| } 
 \hline
 & uri & name \\ 
 \hline
1 & http://dbpedia.org/resource/Beer\_bread & Beer bread\\ 
2 & http://dbpedia.org/resource/Beer\_cheese\_(spread) & Beer cheese\\ 
3 & http://dbpedia.org/resource/Beer\_soup & Beer soup\\
 \hline
\end{tabular} \ \\
\end{adjustwidth}



\aufgabe{3}{\ppp}
\aufgabenteil{a}
\begin{adjustwidth}{20pt}{20pt}
\begin{tt}
$@$prefix rdf: $<$\nolinkurl{http://www.w3.org/1999/02/22-rdf-syntax-ns#}$>$ .\\
$@$prefix rdfs: $<$\nolinkurl{http://www.w3.org/2000/01/rdf-schema##}$>$ .\\
$@$prefix foaf: $<$\nolinkurl{http://xmlns.com/foaf/0.1/}$>$ .\\
$@$prefix uni: $<$\nolinkurl{http://example.org/the-university/}$>$ .\\
\ \\
uni:Angehöriger rdf:type rdfs:Class . \\

uni:Student rdf:type rdfs:Class . \\
uni:Student rdfs:subClassOf uni:Angehöriger .\\
uni:Mitarbeiter rdf:type rdfs:Class .\\
uni:Mitarbeiter rdfs:subClassOf uni:Angehöriger .\\

uni:Professor rdf:type rdfs:Class .\\
uni:Professor rdfs:subClassOf uni:Mitarbeiter .\\
uni:WiMa rdf:type rdfs:Class .\\
uni:WiMa rdfs:subClassOf uni:Mitarbeiter .\\
uni:HiWi rdf:type rdfs:Class .\\
uni:HiWi rdfs:subClassOf uni:Mitarbeiter .\\

uni:Veranstaltung rdf:type rdfs:Class .\\

uni:besucht rdf:type rdf:Property .\\
uni:besucht rdfs:domain uni:Student .\\
uni:besucht rdfs:range uni:Veranstaltung .\\

uni:Vorlesung rdf:type rdfs:Class .\\
uni:Vorlesung rdfs:subClassOf uni:Veranstaltung .\\
uni:Übung rdf:type rdfs:Class .\\
uni:Übung rdfs:subClassOf uni:Veranstaltung .\\
uni:Seminar rdf:type uni:Veranstaltung .\\
uni:Seminar rdfs:subClassOf uni:Veranstaltung .\\

uni:betreutProf rdf:type rdf:Property .\\
uni:betreutProf rdfs:domain uni:Professor .\\
uni:betreutProf rdfs:range uni:Veranstaltung .\\

uni:betreut rdf:type rdf:Property .\\
uni:betreut rdfs:domain uni:Mitarbeiter .\\
uni:betreut rdfs:range uni:Übung .\\
\end{tt}
\end{adjustwidth}
\aufgabenteil{b}
\begin{adjustwidth}{20pt}{20pt}
\begin{tt}
$@$prefix rdf: $<$\nolinkurl{http://www.w3.org/1999/02/22-rdf-syntax-ns#}$>$ .\\
$@$prefix rdfs: $<$\nolinkurl{http://www.w3.org/2000/01/rdf-schema##}$>$ .\\
$@$prefix foaf: $<$\nolinkurl{http://xmlns.com/foaf/0.1/}$>$ .\\
$@$prefix uni: $<$\nolinkurl{http://example.org/the-university/}$>$ .\\
\ \\
uni:stud1 rdf:type uni:Student .\\
uni:stud1 foaf:name "Philipp Hochmann" .\\
uni:prof1 rdf:type uni:Professor .\\
uni:prof1 foaf:name "Matthias Jarke" .\\
uni:wima1 rdf:type uni:WiMa .\\
uni:wima1 foaf:name "Christian Samsel" .\\
uni:hiwi1 rdf:type uni:HiWi .\\
uni:hiwi1 foaf:name "Clemens Löbbert" .\\

uni:dbis rdf:type uni:Vorlesung .\\
uni:dbis\_gü rdf:type uni:Übung .\\

uni:stud1 uni:besucht uni:dbis .\\
uni:stud1 uni:besucht uni:dbisgü .\\
uni:stud1 uni:besucht uni:webscience .\\

uni:prof1 uni:betreutProf uni:dbis .\\
uni:prof1 uni:betreufProf uni:webscience .\\
uni:wima1 uni:betreut uni:dbis\_gü .\\


\end{tt}
\end{adjustwidth}

% --------
\end{document}
