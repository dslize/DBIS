\documentclass[11pt, a4paper]{article}

\usepackage{graphicx} 
\usepackage[utf8]{inputenc}
\usepackage{fancyhdr}
\usepackage{changepage}
\usepackage[onehalfspacing]{setspace}
\usepackage{ragged2e}
\usepackage{ amssymb, amsmath, amsthm, dsfont }
\usepackage[width = 18cm, top = 2.5cm, bottom = 3cm]{geometry}
\usepackage{extarrows}
\usepackage{mathtools}
% --------- Variabel, auf jedem Blatt ändern!
\newcommand{\blattnummer}{4}
\newcommand{\datum}{18. Mai 2017}
	% Punktezahlen & Summe
\newcommand{\p}{6}
\newcommand{\pp}{8}
\newcommand{\ppp}{6}
\newcommand{\pppp}{}
\newcommand{\sump}{20}
% --------- Macros

\newcommand{\myTitleString} {}
\newcommand{\myAuthorString} {}
\newcommand{\mySubTitleString} {}
\newcommand{\myDateString} {}

\newcommand{\myTitle}[1] {\renewcommand {\myTitleString}{#1}}
\newcommand{\mySubTitle}[1] {\renewcommand {\mySubTitleString}{#1}}
\newcommand{\myAuthor}[1] {\renewcommand{\myAuthorString}		{#1}}
\newcommand{\myDate}[1] {\renewcommand{\myDateString}{#1}}

\newcommand{\makeMyTitle}
{
\pagestyle{fancy}
\fancyhead[L]
{
\begin{tabular}{l}
\myTitleString
\\ \mySubTitleString 
\\ \myDateString
\end{tabular}
} 			
\fancyhead[C]{}
\fancyhead[R]{\myAuthorString}
\fancyfoot[C]{\thepage}
}

\setlength{\headheight}{45pt}

\makeatletter
\renewcommand*\env@matrix[1][*\c@MaxMatrixCols c]{%
  \hskip -\arraycolsep
  \let\@ifnextchar\new@ifnextchar
  \array{#1}}
\makeatother

    % args: Aufgabennummer, erreichbare Punkte
\newcommand{\aufgabe}[2] {\section*{Aufgabe \blattnummer.#1 (Punkte:\qquad/#2)}}
\newcommand{\aufgabenteil}[1] {\textbf{(#1)}}
% ---------
%\setlength{\parindent}{0pt}
\begin{document}

\myTitle{\textsc{Datenbanken und Informationssysteme}}
\mySubTitle{Übung \blattnummer}
\myDate{\datum}
\myAuthor
{
\begin{tabular}{l l}
359109, &Michelle Milde\\
356148, &Philipp Hochmann\\
356092, &Daniel Schleiz
\end{tabular}
}
\makeMyTitle

\hfill
\begin{tabular}{|c|c|c|c|}\hline
   \blattnummer.1 & \blattnummer.2 & \blattnummer.3 & $\sum$\\\hline
  	 \qquad/\p & \qquad/\pp & \qquad/\ppp & \qquad/\sump\\\hline % abhängig vom Übungsblatt
 \end{tabular}
\hfill Korrigiert am:\underline{\hspace{3cm}}
\hfill
\vspace*{30pt}


\aufgabe{1}{\p}
\begin{adjustwidth}{20pt}{20pt}
$L \bowtie R \cup (\underset{x \in \mathcal{L}}{\bigtimes} \{ \text{(NULL)} \} \times (R-\Pi_{\mathcal{R}}(L \bowtie R)))$\newline
Wobei $\mathcal{L}$ und $\mathcal{R}$ die Mengen der Attribute von L bzw. R sind.
\end{adjustwidth}


\aufgabe{2}{\pp}
\aufgabenteil{a}
\begin{adjustwidth}{20pt}{20pt}
Gebe die Geburtsdaten aller Patienten an, welche am 27.02.2017 aufgrund einer Alkoholvergeftung behandelt wurden.
\end{adjustwidth}
\aufgabenteil{b}
\begin{adjustwidth}{20pt}{20pt}
Gebe die Lizenznummer und Spezialisierung der Ärzte an, welche bei einer Behandlung ein Medikament des Konzerns Bayer genutzt haben.
\end{adjustwidth}
\aufgabenteil{c}
\begin{adjustwidth}{20pt}{20pt}
Gebe die IDs aller Patienten an, welche im Jahr 2016 vom Arzt John Dorian behandelt wurden.
\end{adjustwidth}
\aufgabenteil{d}
\begin{adjustwidth}{20pt}{20pt}
Gebe die IDs aller Patienten an, welche am oder vor dem 16.5.1987 geboren wurden und nicht aufgrund von Windpocken behandelt werden mussten. 
\end{adjustwidth}
\aufgabenteil{e}
\begin{align*}
\{ &[m1\_name,m2\_name]\ |\ \exists m1\_pzn, m1\_konzern([m1\_pzn,m1\_name,m1\_konzern]\in Medikament \wedge\\ &\exists m2\_pzn, m2\_konzern([m2\_pzn,m2\_name,m2\_konzern]\in Medikament
	\wedge\\ &\nexists p\_id,a\_id,b\_datum,b1\_diagnose,b2\_diagnose([p\_id,a\_id,m1\_pzn,b1\_diagnose]\in behandeln \wedge \\ &[p\_id,a\_id,m2\_pzn,b2\_diagnose] \in behandeln)))\}
\end{align*}



\aufgabe{3}{\ppp}
\aufgabenteil{a}
\begin{adjustwidth}{20pt}{20pt}
\begin{tt}
CREATE TABLE Kunde\\
(KDNummer DECIMAL (10) NOT NULL UNIQUE PRIMARY KEY,\\
Name CHAR (20))
\\ \ \\
CREATE TABLE Mechaniker\\
(eMail CHAR (20) NOT NULL UNIQUE PRIMARY KEY,\\
Name CHAR (20),\\
Adresse VARCHAR (50))
\\ \ \\
CREATE TABLE MechanikerTelNr\\
(eMail CHAR (20) NOT NULL,\\
TelNr DECIMAL (12) NOT NULL,\\
PRIMARY KEY (eMail, TelNr),\\
FOREIGN KEY (eMail) REFERENCES Mechaniker(eMail))
\\ \ \\
CREATE TABLE beauftragt\\
(KDNummer DECIMAL (10),\\
eMail CHAR (20),\\
PRIMARY KEY (KDNummer, eMail),\\
FOREIGN KEY (KDNummer) REFERENCES Kunde(KDNummer),\\
FOREIGN KEY (eMail) REFERENCES Mechaniker(eMail))
\end{tt}
\end{adjustwidth}
\aufgabenteil{b}
\begin{adjustwidth}{20pt}{20pt}
\begin{tt}
ALTER TABLE Mechaniker\\
DROP COLUMN Adresse,\\
ADD Stadt CHAR (20),\\
ADD Strasse CHAR (20),\\
ADD PLZ DECIMAL (5)
\\ \ \\
CREATE TABLE Stammkunde\\
(KDNummer DECIMAL (10) NOT NULL UNIQUE PRIMARY KEY,\\
Dauer DECIMAL (4),\\
FOREIGN KEY (KDNummer) REFERENCES Kunde(KDNummer))
\\ \ \\
CREATE TABLE Neu-Kunde\\
(KDNummer DECIMAL (10) NOT NULL UNIQUE PRIMARY KEY,\\
FOREIGN KEY (KDNummer) REFERENCES Kunde(KDNummer))
\\ \ \\
CREATE TABLE Online-Kunde\\
(KDNummer DECIMAL (10) NOT NULL UNIQUE PRIMARY KEY,\\
FOREIGN KEY (KDNummer) REFERENCES Kunde(KDNummer))
\end{tt}
\end{adjustwidth}
% --------
\end{document}
