\documentclass[11pt, a4paper]{article}

\usepackage{german}
\usepackage{graphicx} 
\usepackage[utf8]{inputenc}
\usepackage{fancyhdr}
\usepackage{changepage}
\usepackage[onehalfspacing]{setspace}
\usepackage{ragged2e}
\usepackage{ amssymb, amsmath, amsthm, dsfont }
\usepackage[width = 18cm, top = 2.5cm, bottom = 3cm]{geometry}
\usepackage{extarrows}
% --------- Variabel, auf jedem Blatt ändern!
\newcommand{\blattnummer}{6}
\newcommand{\datum}{1. Juni 2017}
	% Punktezahlen & Summe
\newcommand{\p}{3}
\newcommand{\pp}{5}
\newcommand{\ppp}{5}
\newcommand{\pppp}{7}
\newcommand{\ppppp}{10}
\newcommand{\sump}{30}
% --------- Macros

\newcommand{\myTitleString} {}
\newcommand{\myAuthorString} {}
\newcommand{\mySubTitleString} {}
\newcommand{\myDateString} {}

\newcommand{\myTitle}[1] {\renewcommand {\myTitleString}{#1}}
\newcommand{\mySubTitle}[1] {\renewcommand {\mySubTitleString}{#1}}
\newcommand{\myAuthor}[1] {\renewcommand{\myAuthorString}		{#1}}
\newcommand{\myDate}[1] {\renewcommand{\myDateString}{#1}}

\newcommand{\makeMyTitle}
{
\pagestyle{fancy}
\fancyhead[L]
{
\begin{tabular}{l}
\myTitleString
\\ \mySubTitleString 
\\ \myDateString
\end{tabular}
} 			
\fancyhead[C]{}
\fancyhead[R]{\myAuthorString}
\fancyfoot[C]{\thepage}
}

\setlength{\headheight}{45pt}

\makeatletter
\renewcommand*\env@matrix[1][*\c@MaxMatrixCols c]{%
  \hskip -\arraycolsep
  \let\@ifnextchar\new@ifnextchar
  \array{#1}}
\makeatother

    % args: Aufgabennummer, erreichbare Punkte
\newcommand{\aufgabe}[2] {\section*{Aufgabe \blattnummer.#1 (Punkte:\qquad/#2)}}
\newcommand{\aufgabenteil}[1] {\textbf{(#1)}}
% ---------
%\setlength{\parindent}{0pt}
\begin{document}

\myTitle{\textsc{Datenbanken und Informationssysteme}}
\mySubTitle{Übung \blattnummer}
\myDate{\datum}
\myAuthor
{
\begin{tabular}{l l}
359109, &Michelle Milde\\
356148, &Philipp Hochmann\\
356092, &Daniel Schleiz
\end{tabular}
}
\makeMyTitle

\hfill
\begin{tabular}{|c|c|c|c|c|c|}\hline
   1 & 2 & 3 & 4 & 5 &$\sum$\\\hline
  	 \qquad/\p & \qquad/\pp & \qquad/\ppp & \qquad/\pppp & \qquad/\ppppp & \qquad/\sump\\\hline % abhängig vom Übungsblatt
 \end{tabular}
\hfill Korrigiert am:\underline{\hspace{3cm}}
\hfill
\vspace*{30pt}


\aufgabe{1}{\p}
\begin{adjustwidth}{20pt}{20pt}
	Es gelte $\alpha \rightarrow \beta\gamma$ und $\gamma \rightarrow \delta\epsilon$. Mit $A_5$ (Dekomposition) gelten dann auch $\alpha \rightarrow \beta$ und $\alpha \rightarrow \gamma$.
	Mit $A_3$ (Transitivität) gilt dann $\alpha \rightarrow \delta\epsilon$. Mit $A_5$ folgt daraus $\alpha \rightarrow \delta$. (Und $\alpha \rightarrow \epsilon$.)\\
	Mit $A_4$ (Vereinigung) gilt dann auch $\alpha \rightarrow \beta\gamma$ und mit nochmaliger Anwendung von $A_4$ folgt $\alpha \rightarrow \beta\gamma\delta$, dies war zu zeigen. Somit
	ist aufgrund der Korrektheit von $A_1 \text{ bis } A_6$ ebenfalls $A_7$ korrekt.
\end{adjustwidth}



\aufgabe{2}{\pp}
\aufgabenteil{a}
\begin{adjustwidth}{20pt}{20pt}

\end{adjustwidth}
\aufgabenteil{b}
\begin{adjustwidth}{20pt}{20pt}

\end{adjustwidth}



\aufgabe{3}{\ppp}
\aufgabenteil{a}
\begin{adjustwidth}{20pt}{20pt}

\end{adjustwidth}
\aufgabenteil{b}
\begin{adjustwidth}{20pt}{20pt}

\end{adjustwidth}



\aufgabe{4}{\pppp}
\aufgabenteil{a}
\begin{adjustwidth}{20pt}{20pt}

\end{adjustwidth}
\aufgabenteil{b}
\begin{adjustwidth}{20pt}{20pt}

\end{adjustwidth}
\aufgabenteil{c}
\begin{adjustwidth}{20pt}{20pt}

\end{adjustwidth}
\aufgabenteil{d}
\begin{adjustwidth}{20pt}{20pt}

\end{adjustwidth}



\aufgabe{5}{\ppppp}
\aufgabenteil{a}
\begin{adjustwidth}{20pt}{20pt}

\end{adjustwidth}
\aufgabenteil{b}
\begin{adjustwidth}{20pt}{20pt}

\end{adjustwidth}
\aufgabenteil{c}
\begin{adjustwidth}{20pt}{20pt}

\end{adjustwidth}



% --------
\end{document}
