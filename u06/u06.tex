\documentclass[11pt, a4paper]{article}

\usepackage{german}
\usepackage{graphicx} 
\usepackage[utf8]{inputenc}
\usepackage{fancyhdr}
\usepackage{changepage}
\usepackage[onehalfspacing]{setspace}
\usepackage{ragged2e}
\usepackage{ amssymb, amsmath, amsthm, dsfont }
\usepackage[width = 18cm, top = 2.5cm, bottom = 3cm]{geometry}
\usepackage{extarrows}
% --------- Variabel, auf jedem Blatt ändern!
\newcommand{\blattnummer}{6}
\newcommand{\datum}{1. Juni 2017}
	% Punktezahlen & Summe
\newcommand{\p}{3}
\newcommand{\pp}{5}
\newcommand{\ppp}{5}
\newcommand{\pppp}{7}
\newcommand{\ppppp}{10}
\newcommand{\sump}{30}
% --------- Macros

\newcommand{\myTitleString} {}
\newcommand{\myAuthorString} {}
\newcommand{\mySubTitleString} {}
\newcommand{\myDateString} {}

\newcommand{\myTitle}[1] {\renewcommand {\myTitleString}{#1}}
\newcommand{\mySubTitle}[1] {\renewcommand {\mySubTitleString}{#1}}
\newcommand{\myAuthor}[1] {\renewcommand{\myAuthorString}		{#1}}
\newcommand{\myDate}[1] {\renewcommand{\myDateString}{#1}}

\newcommand{\makeMyTitle}
{
\pagestyle{fancy}
\fancyhead[L]
{
\begin{tabular}{l}
\myTitleString
\\ \mySubTitleString 
\\ \myDateString
\end{tabular}
} 			
\fancyhead[C]{}
\fancyhead[R]{\myAuthorString}
\fancyfoot[C]{\thepage}
}

\setlength{\headheight}{45pt}

\makeatletter
\renewcommand*\env@matrix[1][*\c@MaxMatrixCols c]{%
  \hskip -\arraycolsep
  \let\@ifnextchar\new@ifnextchar
  \array{#1}}
\makeatother

\newcommand{\rA}{\rightarrow}

    % args: Aufgabennummer, erreichbare Punkte
\newcommand{\aufgabe}[2] {\section*{Aufgabe \blattnummer.#1 (Punkte:\qquad/#2)}}
\newcommand{\aufgabenteil}[1] {\textbf{(#1)}}
% ---------
%\setlength{\parindent}{0pt}
\begin{document}

\myTitle{\textsc{Datenbanken und Informationssysteme}}
\mySubTitle{Übung \blattnummer}
\myDate{\datum}
\myAuthor
{
\begin{tabular}{l l}
359109, &Michelle Milde\\
356148, &Philipp Hochmann\\
356092, &Daniel Schleiz
\end{tabular}
}
\makeMyTitle

\hfill
\begin{tabular}{|c|c|c|c|c|c|}\hline
   1 & 2 & 3 & 4 & 5 &$\sum$\\\hline
  	 \qquad/\p & \qquad/\pp & \qquad/\ppp & \qquad/\pppp & \qquad/\ppppp & \qquad/\sump\\\hline % abhängig vom Übungsblatt
 \end{tabular}
\hfill Korrigiert am:\underline{\hspace{3cm}}
\hfill
\vspace*{30pt}


\aufgabe{1}{\p}
\begin{adjustwidth}{20pt}{20pt}
	Es gelte $\alpha \rightarrow \beta\gamma$ und $\gamma \rightarrow \delta\epsilon$. Mit $A_5$ (Dekomposition) gelten dann auch $\alpha \rightarrow \beta$ und $\alpha \rightarrow \gamma$.
	Mit $A_3$ (Transitivität) gilt dann $\alpha \rightarrow \delta\epsilon$. Mit $A_5$ folgt daraus $\alpha \rightarrow \delta$. (Und $\alpha \rightarrow \epsilon$.)\\
	Mit $A_4$ (Vereinigung) gilt dann auch $\alpha \rightarrow \beta\gamma$ und mit nochmaliger Anwendung von $A_4$ folgt $\alpha \rightarrow \beta\gamma\delta$, dies war zu zeigen. Somit
	ist aufgrund der Korrektheit von $A_1 \text{ bis } A_6$ ebenfalls $A_7$ korrekt.
\end{adjustwidth}



\aufgabe{2}{\pp}
\aufgabenteil{a}
\begin{adjustwidth}{20pt}{20pt}
    (1) Die funktionale Abhängigkeit ist gültig, da alle vorkommenden Attributkombinationen von A und E zu eindeutigen restlichen Attributen führen. (Einzige Dopplung
        bei $a_1,e_4$, jedoch sind die betroffenen Tupel gleich.) \\
    (2) Die funktionale Abhängigkeit gilt nicht. Im vierten Tupel stehen $d_2,e_2$ mit $b_3,f_3$, im achten Tupel aber mit $b_4,f_3$. \\
    (3) Die funktionale Abhängigkeit gilt nicht. Im sechsten Tupel steht $c_2$ mit $d_4, e_5$, im neunten Tupel aber mit $d_5, e_7$.\\
    (4) Die funktionale Abhängigkeit ist gültig, da alle vorkommenden Attributkombinationen von A,C und F zu eindeutigen restlichen Attributen führen. (Einzige Dopplung
        bei $a_1,c_8,f_9$, jedoch sind die betroffenen Tupel gleich.)
\end{adjustwidth}
\aufgabenteil{b}
\begin{adjustwidth}{20pt}{20pt}
	(1) Nein, da $E \rightarrow K$ als funktionale Abhängigkeit nicht impliziert, dass auch $K \rightarrow E$ gilt und sich diese FD nicht aus den Übrigen herleiten lässt (mit Hilfe des
	Armstrong Kalküls, $E$ steht stets nur auf der linken Seite einer FD.) \\
	(2) Ja, da die FD sich aus den gegebenen FDs (mit Hilfe des Armstrong Kalküls) ableiten lässt. \\ \indent$(E \rightarrow BD) \Rightarrow (E \rightarrow B) \Rightarrow (CE \rightarrow CB)
	\Rightarrow (CE \rightarrow B)$. \\
	(3) Ja, da die FD sich aus den gegebenen FDs (mit Hilfe des Armstrong Kalküls) ableiten lässt. \\ \indent$(E \rightarrow IK \text{und} E \rightarrow BD) \Rightarrow
	(E \rightarrow IKBD) \Rightarrow (E \rightarrow BIK) \Rightarrow (AE \rightarrow ABIK, \text{ mit } A \rightarrow CL) \Rightarrow (AE \rightarrow CLBIK) \Rightarrow
	(AEL \rightarrow CLBIK) \Rightarrow (AEL \rightarrow BCIK).$
\end{adjustwidth}



\aufgabe{3}{\ppp}
\aufgabenteil{a}
\begin{adjustwidth}{20pt}{20pt}
	Im Folgenden bezeichne $Erg$ die sich iterativ vergrößernde Attributhülle von $X$. In jedem Schritt wird zur Erweiterung von $Erg$ benutzte FD mittels der auf
	dem Übungsblatt gegebenen Nummer notiert.
	\begin{itemize}
	\item $Erg \leftarrow \{ X\}$. (Entspringt aus trivialer FD $X \rA X$)
	\item $Erg \leftarrow \{ X,A,P\}$. (5) (Da $\{ X\} \subseteq \{ X\}$)
	\item $Erg \leftarrow \{ X,A,P,R\}$. (1) (Da $\{ A\} \subseteq \{ X,A,P\}$)
	\item $Erg \leftarrow \{ X,A,P,R,Y\}$. (4) (Da $\{ P,R\} \subseteq \{ X,A,P,R\}$)
	\end{itemize}
	Da keine weiteren FDs zur Erweiterung von $Erg$ existieren, ist die Attributhülle von $X$ gegeben durch $\{ X,A,P,R,Y\}$.
\end{adjustwidth}
\aufgabenteil{b}
\begin{adjustwidth}{20pt}{20pt}
	Es existieren die Schlüsselkandidaten $\{ X,B,C\}$ und $\{ X,B,Q\}$. Zeige zunächst, dass $\{ X,B,C\}$ und $\{ X,B,Q\}$ Superschlüssel sind:  \\
	Zeige hierzu zunächst, dass $\{ X,B,C\}$ Superschlüssel ist. Dies gilt gdw. die Attributhülle von $\{ X,B,C\}$ der Gesamtmenge der attribute entspricht. Berechne also wieder iterativ
	die Attributhülle.
	\begin{itemize}
	\item $Erg \leftarrow \{ X,A,P,R,Y,B,C\}$. (mit Ergebnis aus \textbf{(a)} und trivialer FD $XBC \rA XBC$)
	\item $Erg \leftarrow \{ X,A,P,R,Y,B,C,Q\}$. (2)
	\item $Erg \leftarrow \{ X,A,P,R,Y,B,C,Q,Z\}$. (3)
	\end{itemize}
	Analog für $\{ X,B,Q\}$:
	\begin{itemize}
	\item $Erg \leftarrow \{ X,A,P,R,Y,B,Q\}$. (mit Ergebnis aus \textbf{(a)} und trivialer FD $XBQ \rA XBQ$)
	\item $Erg \leftarrow \{ X,A,P,R,Y,B,Q,C,Z\}$. (3)
	\end{itemize}
	Da in beiden Fällen die Attributhülle der Menge aller Attribute im Schema entspricht, handelt es sich bei $\{ X,B,C\}$ und $\{ X,B,Q\}$ um Superschlüssel. Nun zur Minimalität
	der beiden Schlüsselkandidaten:\\
	Stelle zunächst fest, dass jeder Schlüsselkandidat $X$ enthalten muss, da $X$ lediglich auf der linken Seite, und damit ein
	Schlüsselkandidat ohne $X$ kein Superschlüssel wäre, da $X$ nie auf der rechten Seite einer FD steht. Gleiches gilt für $B$. Jeder Schlüsselkandidat muss also mindestens
	$B$ und $X$ enthalten. Schaue noch, ob $\{ B,X\}$ ein Superschlüssel ist. Mit Aufgabenteil \textbf{(a)} ergibt sich die Attributhülle $\{ X,A,P,R,Y,B\}$. Ohne $C$ oder $Q$
	wird $Z$ nie ableitbar sein. Somit ist $\{ B,X\}$ kein Superschlüssel. Aus den beiden Beobachtungen folgt nun, dass $\{ X,B,C\}$ und $\{ X,B,Q\}$ minimal sind, da
	jeder Schlüsselkandidat $B$ und $X$ enthalten muss, aber diese nicht genügen.\\ \ \\
	Weiterhin existieren keine anderen Schlüsselkandidaten. Betrachtet man die Attributhülle\\ $\{ X,A,P,R,Y,B\}$ von $XB$, so fehlen lediglich noch die Attribute $C,Q,Z$, um
	Superschlüssel zu sein. Würde man ein Attribut aus der Hülle von $XB$ hinzufügen, um einen Schlüsselkandidaten zu erzeugen, so würde dies sofort dem Minimalitätskriterium
	widersprechen. Somit lässt sich $XB$ nur um $C$, $Q$, oder $Z$ ergänzen. Dabei ist $\{X,B,Z\}$ kein Superschlüssel und somit kein Schlüsselkandidat, weil z.B. $Q$
	nicht in der Attributhülle dieser Menge liegt. Insgesamt folgt nun, dass die gegebenen Kandidaten die Einzigen sind.
\end{adjustwidth}



\aufgabe{4}{\pppp}
\aufgabenteil{a}
\begin{adjustwidth}{20pt}{20pt}
	Das Relationenschema liegt in 1NF vor, da die Aufgabenstellung impliziert, dass keine mengenwertige Attribute vorhanden sind. Es liegt ebenfalls in 2NF vor, denn der einzige
	Schlüsselkandidat ist $\kappa :=\{ B,D,G,F\}$. Jedes der Nichtschlüsselattribute ($A,C,E,H$) ist dabei voll funktional von $\kappa$ abhängig, da man einerseits mithilfe
	des Dekompositionsaxioms und der FD $BDGF \rightarrow ACEH$ die entsprechende FD ableiten lässt, und man dabei jeweils auf der linken Seite kein Attribut weglassen kann. (Die FD
	$H \rightarrow F$ hilft dabei nicht.)\\
	Es liegt ebenfalls in 3NF vor, da die einzigen FDs, welche in $F^+$ liegen und auf der rechten Seite einen Nichtschlüssel haben, der Form $BDGF \rA Z$ sind, für $Z \in \{ A,C,E,H\}$.
	Die Linke Seite ist dabei insbesondere Superschlüssel, da diese bereits Schlüsselkandidat ist und damit $X$ voll funktional abhängig von $\kappa$.
\end{adjustwidth}
\aufgabenteil{b}
\begin{adjustwidth}{20pt}{20pt}
	Das Relationenschema liegt in 1NF vor, da die Aufgabenstellung impliziert, dass keine mengenwertige Attribute vorhanden sind. Es liegt ebenfalls in 2NF vor. (Einziger
	Schlüsselkandidat $\kappa :=\{ A,D,H\}$, gleiche Argumentation wie in \textbf{(a)}.) Auch in 3NF, gleiche Argumentation wie in \textbf{(a)} und offensichtlich
	keine FDs zwischen Nichtschlüsseln.
\end{adjustwidth}
\aufgabenteil{c}
\begin{adjustwidth}{20pt}{20pt}
	Das Relationenschema liegt in 1NF vor, da die Aufgabenstellung impliziert, dass keine mengenwertige Attribute vorhanden sind. Auch in 2NF, da einziger Schlüsselkandidat mit
	$\kappa := \{ C,D,E,G\}$. Jedes der Nichtschlüsselattribute ($A,B,F,H$) ist dabei voll funktional von $\kappa$ abhängig, da mittels Transitivitätsaxiom alle FDs $CDEG \rA
	Z$ für $Z \in \{ A,B,F,H\}$ ableitbar sind und diese jeweils nicht linksreduzierbar sind.\\
	Es liegt nicht in 3NF vor, da FDs zwischen Nichtschlüsselattributen vorliegen (z.B. $F \rA H$). Wandle also um mittels Synthesealgorithmus:\\
	Für die kanonische Überdeckung muss nichts getan werden. (Weder links- noch rechtsreduzierbar, nichts vereinigbar.) Generiere nun für jede FD ein Relationenschema:
	\begin{itemize}
	\item $R_1=(\{ C,D,E,F,G\}, \{ CDEG \rA F\})$
	\item $R_2=(\{ F,H\}, \{ F \rA H\})$
	\item $R_3=(\{ A,H\}, \{ H \rA A\})$
	\item $R_4=(\{ A,B\}, \{ A \rA B\})$
	\end{itemize}
	Nun enhält $R_1$ den Schlüsselkandidaten $\kappa$. Da zudem keine Attributmenge eines enstandenen Schemas in einem anderen Schema liegt, ist man hier fertig.
\end{adjustwidth}
\aufgabenteil{d}
\begin{adjustwidth}{20pt}{20pt}
	Das Relationenschema liegt in 1NF vor, da die Aufgabenstellung impliziert, dass keine mengenwertige Attribute vorhanden sind. Stelle
	zunächst fest, dass der einzige Schlüsselkandidat $\kappa := \{ E,G\}$ ist: $\kappa$ ist Superschlüssel da die Attributhülle von $EG$ gerade $X$ ist (wegen $G \rA B$ und
	$E \rA AF$ ist die Attributhülle zunächst $\{ E,G, B, A, F\}$, im nächsten Schritt mit $F \rA CDH$ ergibt sich $X$). $\kappa$ ist außerdem minimal, da man bei Weglassen
	eines der Attribute ($E$ oder $G$) jeweils abgeleitete FDs verliert, es gilt z.B. weder $G \rA A$, noch $E \rA B$.\\
	Das Schema liegt nicht in 2NF vor, da aufgrund gerade genannter Beispiele ein paar Attribute nicht voll funktional abhängig von $\kappa$ sind. Überführe nun in 2NF,
	es ergeben sich folgende Schemata:
	\begin{itemize}
	\item $R_1=(\{ E,A,F,C,D,H\}, \{ E \rA AF, F \rA CDH\})$
	\item $R_2=(\{ G,B\}, \{ G \rA B\})$
	\item $R_3=(\{ E,G\}, \{\})$ \ (Hier FD einfügen? TODO)
	\end{itemize}
	Nun sind $R_2$ und $R_3$ in 3NF, da keine FDs zwischen Nichtschlüsselattributen vorhanden sind. (In $R_2$ ist offensichtlich $G$ Schlüssel, da sonst keine FDs auch Superschlüssel.)
	$R_1$ ist aber nicht in 3NF, da mit $F \rA CDH$ eine FD zwischen Nichtschlüsselattributen besteht, da E der Superschlüssel ist. (Ergibt sich mit Transitivitätsaxiom.)
	Wandle das Schema also in 3NF um:\\
	Die kanonische Überdeckung ist bereits gegeben. Generiere also für jede FD ein Relationenschema:
	\begin{itemize}
	\item $R_{11} = (\{ E,A,F\}, \{ E \rA AF\})$
	\item $R_{12} = (\{ F,C,D,H\}, \{ F \rA CDH\})$
	\end{itemize}
	Da nun $R_{11}$ den Schlüsselkandidaten $E$ enthält und keine Attributmenge eines enstandenen Schemas in einem anderen Schema liegt, ist man fertig.
\end{adjustwidth}



\aufgabe{5}{\ppppp}
\aufgabenteil{a}
\begin{adjustwidth}{20pt}{20pt}
$R$ ist nicht in BCNF, da z.B. das Attribut $A$ kein Superschlüssel ist, es aber die FD $A \rA C$ gibt. Die Attributhülle von $A$ ist $\{A, C\}$, was nicht der Menge aller Attribute aus $R$ entspricht.
\end{adjustwidth}
\aufgabenteil{b}
\begin{adjustwidth}{20pt}{20pt}

\begin{enumerate}
\item $Z$ enthalte:\\
	$R_1=(\{A, B, C, D, E, F, G, H, I \},$\\
	$\{A \rA C, B \rA A, DE \rA I, CI \rA GH, EI \rA AB, DB \rA C\})$\\

\item Dekomposition entlang $A \rA C$ in $R_1$:\\
	$\begin{aligned}
	R_{12}&=(\{A, B, C, D, E, G, H, I\}, \{B \rA A, DE \rA I, EI \rA AB\})\\
	R_{11}&=(\{A, C\}, \{A \rA C\})
	\end{aligned}$\\
	
\item Dekomposition entlang $B \rA A$ in $R_{12}$:\\
	$\begin{aligned}
	R_{122}&=(\{B, D, E, G, H, I\}, \{DE \rA I\})\\
	R_{121}&=(\{B, A\}, \{B \rA A\})\\
	R_{11}&=(\{A, C\}, \{A \rA C\})
	\end{aligned}$\\
	
\item Dekomposition entlang $DE \rA I$ in $R_{122}$:\\
	$\begin{aligned}
	R_{1222}&=(\{B, D, E, G, H\}, \emptyset)\\
	R_{1221}&=(\{D, E, I\}, \{DE \rA I\})\\
	R_{121}&=(\{B, A\}, \{B \rA A\})\\
	R_{11}&=(\{A, C\}, \{A \rA C\})
	\end{aligned}$\\

\end{enumerate}
Damit ist jede Relation in BCNF.
\end{adjustwidth}
\aufgabenteil{c}
\begin{adjustwidth}{20pt}{20pt}
Die Zerlegung war nicht abhängigkeitserhaltend, da z.B. die FD $EI \rA AB$ in keiner der aus der Zerlegung resultierenden Relationen besteht, jedoch in der Ursprungsrelation $R_1$ bestand.
\end{adjustwidth}



% --------
\end{document}
